\documentclass[10pt]{article}

\usepackage{mcclanahan}

\usepackage[backend=biber]{biblatex}
\addbibresource{parsing-and-computation.bib}

\newcommand{\todocite}[1]{\footnote{cite: #1}}
\newcommand{\todo}[1]{\textbf{?} \footnote{TODO: #1}}

\title{Efficiently Inferring Non-hierarchical Structure in Parsing and Computation}
\date{\today}
\author{Danny McClanahan}

\begin{document}
\maketitle

\textit{Abstract:} This paper proposes to revise the general definition of a ``grammar''\todocite{grammar defn} as well as ``parsing''\todocite{parsing defn}. A meta-language named ``Simultaneous Productions'' (or ``S.P'') is proposed for specifying a parseable grammar (similar in intent to EBNF\todocite{EBNF}). This meta-language is shown to be able to represent recursively enumerable languages\todocite{RecEnum defn}, and several favorable properties of this meta-language across various parsing use cases are discussed.

An actual ``evaluation method'' (or ``parsing algorithm'') is then introduced for the proposed S.P. grammar model, which shares some similarities with the CYK algorithm \todocite{CYK algorithm!}.This method is shown to terminate in \todo{parsing runtime?} time across \todo{parsing inputs?} inputs. This method is then demonstrated to have the peculiar property of directly parameterizing context-sensitivity \todocite{context-sensitivity defn} \todo{is this peculiar?}. The practical and theoretical ramifications of this result are discussed and speculated on.

\textit{This paper will frequently refer to free software provided at \cite{repo-sp} which implements all of the theoretical mechanisms discussed.}

\newpage
\tableofcontents
\newpage

\section{Motivation}
\label{sec:motivation}

\section{Meta-Language for Grammar Specification}
\label{sec:meta-language-for-grammar-specification}

\section{Fully General Parse Method}
\label{sec:fully-general-parse-method}

\section{Re-parameterization of the Chomsky Hierarchy}
\label{sec:re-parameterization-of-the-chomsky-hierarchy}

\textit{TODO: describe the hierarchy as being better described by derivatives (?) of a function corresponding to stack depth (!!!) and link it to approximation difficulty!}\todocite{approximation difficulty}

\printbibliography
\end{document}
